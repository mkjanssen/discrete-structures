\documentclass[a4paper,12pt]{letter}

\usepackage{hyperref}

% Some of the article customisations are relevant for this class

%\name{} % To be used for the return address on the envelope
%\signature{} % Goes after the closing (ie at the end of the letter, with space for a signature)
%\address{Address \\ of \\ Sender}
% Alternatively, these may be set on an individual basis within each letter environment.

%\makelabels % this command prints envelope labels on the final page of the document

\begin{document}
%\begin{letter}{Name and \\ Address \\ of \\ Receiver}

%\opening{} % eg Hello.

\begin{center}
{\bf Installing \LaTeX}

\vspace{-2mm}

\end{center}


This document contains 3 ways to access \LaTeX:
	\begin{enumerate}
	\item Download to a Mac (page 2)
	\item Download to a PC (page 3)
	\item Use an online version (page 4)
	\end{enumerate}

Having a version downloaded to your computer is nice in case you do not have access to the internet, but students have found Overleaf (the online version) to be easy to use as well.

\newpage

\begin{center}
{\bf Mac OS X Instructions}
\end{center}

If at any point, your computer says an app cannot be opened due to being from an unidentified developer, try right clicking and selecting open.  This should give you a pop up that allows you to open the program.


\begin{center}
{\bf Latex - Mac OS X}
\end{center}

\begin{enumerate}
\item Download the installation file at the link below %(when you click the link it will start downloading automatically).
\begin{center}
\url{https://tug.org/mactex/mactex-download.html}
\end{center}
This file is rather large, so I recommended you download it on a relatively fast internet connection.  
\item Open the downloaded ``MacTeX.pkg'' file to launch the LaTeX installer.  
\item Follow the onscreen instructions for installation.  All of the default settings should be fine.  The installation may take a while, but you can feel free to do other things while it installs.  
\item Once the installation finishes, you should test your installation.  Download and open the test file from Blackboard (under class worksheets - LaTeX) or go here: \url{https://drive.google.com/open?id=1Fj3Y_jAgGiqVfp1fUIgQeq5YPN-Aw2rZ}
It should open in a program called TeXShop (which you just installed).  If not, right click and select ``open with" and choose TeXShop.
\item Go to the Typeset menu, and select Typeset.  This should briefly display some output, and then open a pdf file with the header ``Sample \LaTeX\,  File''.  If it does, you are good to go!  
\end{enumerate}

\pagebreak

\begin{center}
{\bf Installing \LaTeX}

\vspace{-2mm}

\end{center}

\begin{center}
{\bf Windows Instructions}
\end{center}


\begin{center}
{\bf Latex - Windows}
\end{center}

\begin{enumerate}
\item Download the MikTex installer here: 
\begin{center}
\url{http://miktex.org/download}
\end{center}
\item Launch the installer, and follow the onscreen instructions.  
\item Once the installation finishes, you should test your installation.  Download and open the test file from Blackboard (under class worksheets - LaTeX) or go here: \url{https://drive.google.com/open?id=1Fj3Y_jAgGiqVfp1fUIgQeq5YPN-Aw2rZ}
It should open in a program called Texworks (which you just installed).  
%\item Before doing anything else, go to the File menu, and select 
\item Go up to the Typeset menu, and select Typeset.  As it is compiling, it may prompt you to download additional packages.  Just click OK for each dialog.  In the end a PDF should open with the header ```I Heart \LaTeX"
\end{enumerate}


\newpage

\begin{center}
{\bf Using an Online Version of \LaTeX}
\end{center}

\begin{enumerate}
\item Go to \url{https://www.overleaf.com} and create an account. You do NOT need a paid version.
\item  Once you've logged in, you should create a new project.  This will generate a \LaTeX file.  From this you should be able to create a PDF by clicking the green ``Reompile" button on the right half of the split screen.
\item You should test uploading a file. 
	\begin{enumerate}
	\item Download the test file from Blackboard or here Download and open the test file from Blackboard (under class worksheets - LaTeX) or go here: \url{https://drive.google.com/open?id=1Fj3Y_jAgGiqVfp1fUIgQeq5YPN-Aw2rZ}
	\item On Overleaf on the top left under the menu there are three icons. Select the third icon.
	\item Upload the test file from Blackboard.
	\item Click the arrow next to main.tex and delete that file. Select TestFile.tex and then click the green "Recompile" button. You should see a document that has title ``I Heart \LaTeX"
	\end{enumerate}
\end{enumerate}


\end{document}
