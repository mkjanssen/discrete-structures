\documentclass{article}  %Need this.

\usepackage{amsmath,amsthm,amssymb}


\newtheorem*{thm}{Theorem}
\newtheorem*{cnj}{Conjecture}
\newtheorem*{lem}{Lemma}
\newtheorem*{cor}{Corollary}
\newtheorem*{prop}{Proposition}

\newcommand{\N}{\mathbb{N}}
\newcommand{\Z}{\mathbb{Z}}
\newcommand{\R}{\mathbb{R}}





\title{Practice}
\author{--Name --}
\date{}

\begin{document}
\maketitle  %This will add the title, author, and date located above

\begin{thm}
If $x$ and $y$ are odd integers then $x+y$ is an even integer.
\end{thm}

\begin{proof}
We assume that $x$ and $y$ are odd integers and will prove that $x+y$ is an even integer. Since $x$ and $y$ are odd, there exist integers $m$ and $n$ such that $x=2m+1$ and $y=2n+1$.  By substitution and algebra we obtain
\begin{align*}
x+y &= 2m+1 + 2n + 1\\
&= 2m+2n+2\\
&=2(m+n+1).
\end{align*}
Define $q=m+n+1$.  Since $m$ and $n$ are integers and the integers are closed under addition, we conclude that $q$ is an integer. Since $x+y=2q$ for the integer $q$ we conclude that $x+y$ is an even integer.
\end{proof}


\vspace{.15in}

\section*{Challenge Typing}


Suppose that $f:(-1,1)\to \R$ and $f$ is differentiable at $0$. Let sequences $(\alpha_n)_{n\geq1}$ and $(\beta_n)_{n\geq1}$ satisify $-1<\alpha_n<\beta_n<1$ for all $n\geq 1$ and $\displaystyle{\lim_{n\to\infty}} \alpha_n = \displaystyle{\lim_{n\to\infty}} \beta_n = 0$. Set 
	
		$$\lambda_n = \frac{f(\beta_n) -f(\alpha_n)}{\beta_n-\alpha_n} . $$

\newpage

\begin{thm}
The set $\{ x\in\Z : |x-2.5| = 2 \}$ is the empty set.
\end{thm}

\begin{proof}
Let $y$ be an integer such that $y \in \{ x\in\Z : |x-2.5| = 2 \}$  Then $y\in \mathbb{Z}$ and $|y-2.5| = 2$.  Since $|y-2.5| = 2$ then $y = 4.5$ or $y=-.5$.  But then $y$ is not an integer.  Therefore the set $\{ x\in\Z : |x-2.5| = 2 \}$ has no elements and
\[\{ x\in\Z : |x-2.5| = 2 \} = \emptyset.\]
\end{proof}

\begin{thm}
There exist two positive irrational numbers $s$ and $t$ such that $s^t$ is rational.
\end{thm}
\begin{proof}
We will consider two cases.  For the first case, suppose that $\sqrt{2}^{\sqrt{2}}$ is rational.  Then we may take $s= t = \sqrt{2}$.  For the second case, suppose that $\sqrt{2}^{\sqrt{2}}$ is irrational.  Let $s= \sqrt{2}^{\sqrt{2}}$ and $t=\sqrt{2}$.  Then 
$$\left(\sqrt{2}^{\sqrt{2}}\right)^{\sqrt{2}} = \left(\sqrt{2}\right)^2 = 2.$$
Since $2$ is rational, $s^t$ is rational.  Therefore, there exists irrational numbers $s$ and $t$ such that $s^t$ is rational.
\end{proof}

\begin{thm}
Let $n$ be a natural number.  Then $$\sum_{k=1}^n k = \frac{n(n+1)}{2}.$$
\end{thm}

Consider the following matrix,
$$\left( \begin{array}{ccc}
1 &2 &3\\
4 &5 &6\\
7 &8 &9
\end{array}\right)$$



\end{document}