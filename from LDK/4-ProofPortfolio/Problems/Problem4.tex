\documentclass{article}  %Need this.

\usepackage{amsmath,amsthm,amssymb}

\usepackage[margin=1in]{geometry}


\newtheorem*{thm}{Theorem}
\newtheorem*{cnj}{Conjecture}
\newtheorem*{lem}{Lemma}
\newtheorem*{cor}{Corollary}
\newtheorem*{prop}{Proposition}

\newcommand{\N}{\mathbb{N}}
\newcommand{\Z}{\mathbb{Z}}
\newcommand{\R}{\mathbb{R}}





\title{Proof Portfolio Problem 4}
\author{}
\date{}

\begin{document}
\maketitle

Chose one problem from each numbered group.  For example, choose only one of 4A, 4B, and 4C.\\




%Using proof by contrapositive	

%\noindent\textbf{Conjecture 4A.}\footnote{You can find the definition of composite number on page 78 of your text.}  For all natural numbers $n$, if $3$ does not divide $n^2+2$ then $n$ is a composite number or $n=3$.\\ This one is proof by cases....


\noindent\textbf{Conjecture 4A.}  For all real numbers $b$, if $b>4$ then $\dfrac{b^2-3b+16}{2b+2}>2$.\\

\noindent\textbf{Conjecture 4B.}  For all integers $z$, if $z^3-4z$ is even then $z$ is even.\\

\noindent\textbf{Conjecture 4C.} For all integers $x$ and $y$ if $3\nmid x+y$ then $3\nmid x$ or $3\nmid y$.




\end{document}