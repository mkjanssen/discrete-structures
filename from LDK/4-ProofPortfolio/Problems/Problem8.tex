\documentclass{article}  %Need this.

\usepackage{amsmath,amsthm,amssymb}

\usepackage[margin=1in]{geometry}


\newtheorem*{thm}{Theorem}
\newtheorem*{cnj}{Conjecture}
\newtheorem*{lem}{Lemma}
\newtheorem*{cor}{Corollary}
\newtheorem*{prop}{Proposition}

\newcommand{\N}{\mathbb{N}}
\newcommand{\Z}{\mathbb{Z}}
\newcommand{\R}{\mathbb{R}}





\title{Proof Portfolio Problem 8}
\author{}
\date{}

\begin{document}
\maketitle

As a reminder, you should pick only one of the following problems. Remember to start ASAP and see me if you need help.

\emph{The initial deadline for Problems 5-8 is Monday, March 23 (11:59PM). The final deadline is Friday, March 30 (11:59PM).}

%Sets
\noindent\textbf{Conjecture 8A.}\footnote{The symbol $\times$ is defined on page 256 of your text.  For two sets $X$ and $Y$, $X \times Y = \{(x,y) \mid x \in X \text{ and } y\in Y.\}$ Think of ordered pairs, like you're graphing on the Cartesian plane.} If $A,B,$ and $C$ are subsets of some universal set $U$ then
 \[A\times (B\cup C) = (A\times B) \cup(A\times C).\]\\
 


\noindent\textbf{Conjecture 8B.} Let $X = \{x \in\mathbb{Z} : x \equiv 2 \pmod{6} \}$ and $Y = \{y\in\mathbb{Z} : 3 \mid y-5\}$.  Prove that one of these sets is a proper subset of the other (stating your result as a theorem).\\




\section*{Some LaTeX Notes:}
For 8A:
\begin{verbatim}
$A\times (B-C) = (A\times B) - (A\times C)$
\end{verbatim}


\noindent For 8B:  \begin{verbatim}
$X = \{x \in\mathbb{Z} : x \equiv 2 \pmod{6} \}$ \end{verbatim}
and 

\begin{verbatim}$Y = \{y\in\mathbb{Z} : 3 \mid y-5\}$\end{verbatim}
(The \begin{verbatim} \ \end{verbatim} makes the set braces appear.) You could also use \begin{verbatim} \Z \end{verbatim} if you are using my LaTeX file (instead of \begin{verbatim}\mathbb{Z} \end{verbatim}) .)


\end{document}