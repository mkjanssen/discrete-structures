\documentclass[10pt]{article}
\usepackage[margin=1in]{geometry}
\usepackage{nopageno}
\usepackage{hyperref}
\usepackage{multicol}


\usepackage{amsmath,amssymb,graphicx,amsthm}

\newcommand{\R}{\mathbb{R}}
\newcommand{\Z}{\mathbb{Z}}
\newcommand{\E}{\mathbb{E}}
\newcommand{\Q}{\mathbb{Q}}
\newcommand{\N}{\mathbb{N}}
\newcommand{\cM}{\mathcal{M}}
\usepackage{xcolor}
\newcommand{\blue}{\textcolor{blue}}

\usepackage{versions}
\includeversion{solution}

\newcommand{\bs}{\begin{solution}}
\newcommand{\es}{\end{solution}}

\newtheorem{thm}{Theorem}


\begin{document}
\vspace{-1.2in}
\begin{center} \textbf{\Large{Skill Mastery Quiz 7}} \\
Communicating in Math (MTH 210-01)\\
Winter 2020
\end{center}



\noindent Name: 




\begin{itemize}
	




\item[P1-3] Consider the following statement:
		\begin{center}
		For all integers $a$ and $b$, if $a\neq 0$ and $a$ does not divide $b$, then $ax^3+bx+(b+a)=0$ does not have a solution that is a natural number.
		\end{center}
	State what you would assume in a direct proof. 
	
	\bs\textcolor{blue}{Assume that $a$ and $b$ are integers with $a\neq 0$ and that $a$ does not divide $b$. (Basically, we assume the hypothesis.)}\end{solution}
	
	\vfill
	
	
	State what you would assume in a proof by contradiction.
	
	\bs\textcolor{blue}{Assume that there exist integers $a$ and $b$ such that $a\neq 0$ and $a$ does not divide $b$ and $ax^3+bx+(b+a)=0$ does have a solution that is a natural number. (Basically, we assume the negation.)}\end{solution}
\vfill

\newpage



\item [P2-1]  For which of the following situations is it more appropriate to use induction. Explain.
		\begin{enumerate}
		\item For all $a\in\Z$ the equation $ax^3+ax + a = 0$ does not have a solution that is a natural number.
		\item For each natural number $n$, $3$ divides $4^n-1$.
		\end{enumerate}
	Circle one and explain why you chose that.
	
	\bs \blue{We choose the second statement because it starts "for each natural number $n$". Induction works when you have a base case and a way to get from one step to the next.}\end{solution}
	\vspace{2in}
	
	For the statement you chose, state what your steps would be in a proof by induction.
	
	\bs \blue{ We let $P(n)$ be the predicate $3\mid 4^{n} - 1$. Then we prove $P(1)$ which is $3\mid 4^1 - 1$. Next we let $k\in\N$ and assume $P(k)$ or that $3\mid 4^k - 1$. Finally using that we prove that $3\mid 4^{k+1} -1 $.}\end{solution}
\vfill


\end{itemize}	

\end{document}
