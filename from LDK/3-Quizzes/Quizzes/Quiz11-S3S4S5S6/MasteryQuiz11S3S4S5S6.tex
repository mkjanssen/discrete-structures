\documentclass[10pt]{article}
\usepackage[margin=1in]{geometry}
\usepackage{nopageno}
\usepackage{hyperref}
\usepackage{multicol}


\usepackage{amsmath,amssymb,graphicx,amsthm}

\newcommand{\R}{\mathbb{R}}
\newcommand{\Z}{\mathbb{Z}}
\newcommand{\E}{\mathbb{E}}
\newcommand{\Q}{\mathbb{Q}}
\newcommand{\N}{\mathbb{N}}
\newcommand{\cM}{\mathcal{M}}
\usepackage{xcolor}
\newcommand{\blue}{\textcolor{blue}}

\usepackage{versions}
\excludeversion{solution}

\newcommand{\bs}{\begin{solution}}
\newcommand{\es}{\end{solution}}

\newtheorem{thm}{Theorem}

\usepackage{enumitem}


\begin{document}
\vspace{-1.2in}
\begin{center} \textbf{\Large{Skill Mastery Quiz 11}} \\
Communicating in Math (MTH 210-01)\\
Winter 2020
\end{center}



\noindent Name: 




\begin{itemize}
	









\item[S3-3]  Let $f: \Z\to \Z$ be defined by $f(m) = m+3$.
		\begin{enumerate}
		\item State the domain, codomain, and range of $f$. (Clearly state which one is which.)
		\bs\blue{The domain is $\Z$ and the codomain is $\Z$, note these are both given in the definition of the function. The range in this case is also $\Z$ since the outputs will be $\{\dots, -1+3, 0+3,1+3,2+3,\dots\} = \{\dots, 2,3,4,5,\dots\}$.}\end{solution}
		\vspace{1in}
		\item Find the image(s) of $-1$ under $f$.
		\bs\blue{Since $f(-1) = -1+3 = 2$, so the image of $-1$ under $f$ is $2$}\end{solution}
		\vspace{1in}
		\item Find the preimage(s) of $4$.
		\bs\blue{Solve $f(x) = 4$ and take the ones that are in the domain. In this case $4=m+3$ gives only one preimage, $1$.}\end{solution}
		\vspace{1in}
		\end{enumerate} 


\item[S4-2] Let $A$ and $B$ be sets. Carefully complete the definitions of the following terms. (Note: ``no collisions" and ``range=codomain" are helpful ways to think about these, but they are NOT the definitions.)
		\begin{enumerate}
		\item  A function $f: A \to B$ is injective provided that...
		\bs\blue{for all $x,y\in A$ if $x\neq y$ then $f(x) \neq f(y)$}\end{solution}
		\vspace{1in}
		\item A function $f: A\to B$ is surjective provided that...
		\bs \blue{for all $y\in B$, there exists $x\in A$ such that $f(x)=y$}\end{solution}
		\vspace{1in}
		\item A function $f: A\to B$ is bijective provided that...
		\bs\blue{$f$ is both injective and surjective}\end{solution}
		\vspace{1in}
		\end{enumerate}


\item[S6-2] Let $x,y\in\Z$ and $n\in\N$. State the definitions of the following:
\begin{enumerate}
\item $x\mid y$ (for nonzero $r$) \bs\blue{there exists an integer $k$ such that $xk=y$.}\end{solution}
\vspace{.5in}
\item  $x \equiv y\pmod{n}$.  
\bs\blue{$n\mid x-y$}\end{solution}
\vspace{.5in}
\end{enumerate}

Give an example of integers $x$ and $y$ such that $x\nmid y $ and $y<0$.
\bs\blue{$x = 10$ and $y=-3$ then for all integers $k$, $10k \neq -3$. There are lots of answers to this question though!}\end{solution}
\vspace{1in}

\item[S5-1] For all $a,b\in \Z$ say $a\sim b$ if and only if $a\mid b$.  Is $\sim$ an equivalence relation? Explain.
\bs \blue{This is not an equivalence relation. To justify you only have to explain one of not being reflexive or not being symmetric. This relation is not symmetric since $0\in\Z$ and $0\nmid 0$. It is also not symmetric since, for example, $2,6\in\Z$ and $2\mid 6$, but $6\nmid 2$.} \end{solution}

\end{itemize}

\end{document}
