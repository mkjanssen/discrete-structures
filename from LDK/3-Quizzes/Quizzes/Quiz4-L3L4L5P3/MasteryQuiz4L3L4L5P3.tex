\documentclass[10pt]{article}
\usepackage[margin=1in]{geometry}
\usepackage{nopageno}
\usepackage{hyperref}
\usepackage{multicol}


\usepackage{amsmath,amssymb,graphicx}

\newcommand{\R}{\mathbb{R}}
\newcommand{\Z}{\mathbb{Z}}
\newcommand{\E}{\mathbb{E}}
\newcommand{\Q}{\mathbb{Q}}
\newcommand{\N}{\mathbb{N}}
\newcommand{\cM}{\mathcal{M}}
\usepackage{xcolor}
\newcommand{\blue}{\textcolor{blue}}

\usepackage{versions}
\includeversion{solution}

\newcommand{\bs}{\begin{solution}}
\newcommand{\es}{\end{solution}}



\begin{document}
\vspace{-1.2in}
\begin{center} \textbf{\Large{Skill Mastery Quiz 4}} \\
Communicating in Math (MTH 210-01)\\
Winter 2020
\end{center}



\noindent Name: 




\begin{itemize}
	
\item[L3-3] Construct a truth table for $P \rightarrow (Q\vee R)$.

\bs
\begin{center}
\begin{tabular}{c|c|c|c|c}
$P$ &$Q$ &$R$ &$Q\vee R$ &$P\rightarrow (Q\vee R)$\\
\hline
T &T &T  &T &T \\
T &F &T  &T &T \\
F &T &T  &T &T \\
F &F &T  &T &T \\
T &T &F &T&T \\
T &F &F &F&F \\
F &T &F &T &T \\
F &F &F &F &T\\
\end{tabular}
\end{center}
\end{solution}
\vfill
\vfill
\vfill


\item[L4-3]  Write the set $\{\dots, -5,-3,1,1,3,5,\dots\}$ using set builder notation.

\bs \textcolor{blue}{As usual there are many ways to do this. One way is $\{x\in\Z \mid x=2n+1 \text{ for some } n\in\Z\}$.}
\end{solution}

\vfill

\newpage

\item[L5-2] Write a useful negation of the following statement:
		\begin{center}
		There exists $n\in\N$ such that if $a\in\N$ then $\frac{1}{n}<a$.
		\end{center}
		Useful negations don't start with ``It is not true that..." and avoid the word not in cases where it could be replaced (e.g., don't use ``not even").

\bs \textcolor{blue}{A negation is: ``for all $n\in\mathbb{N}$, $a\in\mathbb{N}$ and $\frac{1}{n}\geq a$." A negation that is slightly better worded is ``for all $n\in\mathbb{N}$ there exists $a\in\mathbb{N}$ such that $\frac{1}{n}\geq a$."}
\end{solution}

\vfill

\item[P3-1]  The following statement is incorrect:
		\begin{center}
		If $n$ is an integer then $n^2\equiv 1\pmod{3}$.
		\end{center}
			Show the statement is false using a counterexample. You should clearly explain why the counterexample you found shows the statement is false.


\bs \textcolor{blue}{This statement is false and there are many counterexamples. For example, consider $n=0$. Note $n$ is an integer and $n^2 = 0$. However, $3\nmid 0-1$ (because there is no integer $x$ such that $3x=-1$, and therefore $n^2\not\equiv 1\pmod{3}$. Thus we have found an $n$ that makes the hypothesis true and the conclusion false, making the statement false.}\end{solution}
\vfill
\end{itemize}
	

\end{document}
