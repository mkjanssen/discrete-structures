\documentclass[10pt]{article}
\usepackage[margin=1in]{geometry}
\usepackage{nopageno}
\usepackage{hyperref}
\usepackage{multicol}


\usepackage{amsmath,amssymb,graphicx}

\newcommand{\R}{\mathbb{R}}
\newcommand{\Z}{\mathbb{Z}}
\newcommand{\E}{\mathbb{E}}
\newcommand{\Q}{\mathbb{Q}}
\newcommand{\cM}{\mathcal{M}}
\usepackage{xcolor}
\newcommand{\blue}{\textcolor{blue}}

\usepackage{versions}
\includeversion{solution}

\newcommand{\bs}{\begin{solution}}
\newcommand{\es}{\end{solution}}

\usepackage{enumitem}

\begin{document}
\vspace{-1.2in}
\begin{center} \textbf{\Large{Skill Mastery Quiz - Dylan L. L2, P2, S2}} \\
Communicating in Math (MTH 210-01)\\
Winter 2020
\end{center}






\begin{itemize}
	
\item[L2-token] State the definition of even integer precisely:

An integer $n$ is even provided that...



\vfill
Outline a proof that if $x$ is even and $y$ is odd then $xy$ is even. (Make sure to include key details - like what things are integers.)
 

\vfill
\vfill
\vfill
\vfill

\newpage

\item[P2-token] 	Which of the following situations is it more appropriate to use induction?
\begin{enumerate}
\item For all integers $a$, $a^3 \equiv a \pmod{3}$.
\item For each natural number $n$, 
$$1+3+\cdots + (2n+1)  =n^2.$$
\end{enumerate}

\begin{itemize}
\item Explain your choice.
\vspace{.75in}

\item Outline the steps you would do in a proof by induction.
\vspace{.5in}
\end{itemize}

\vfill

\item[S2-token] Let $U = \{x\in\Z: 1\leq x\leq 10\}$ be the universal set. Let $A = \{1,2,4,8\}$ and $B = \{4,7,9\}$.
	\begin{enumerate}
	\item Find $A\cap B$
	\vspace{.4in}
	\item Find $A\cup B$.
	\vspace{.4in}
	\item Find $A-B$.
	\vspace{.4in}
	\item Find $A^c$.
	\vspace{.4in}
	\end{enumerate}
\end{itemize}
	

\end{document}
