\documentclass[10pt]{article}
\usepackage[margin=1in]{geometry}
\usepackage{nopageno}
\usepackage{hyperref}
\usepackage{multicol}


\usepackage{amsmath,amssymb,graphicx}

\newcommand{\R}{\mathbb{R}}
\newcommand{\Z}{\mathbb{Z}}
\newcommand{\E}{\mathbb{E}}
\newcommand{\Q}{\mathbb{Q}}
\newcommand{\cM}{\mathcal{M}}
\usepackage{xcolor}
\newcommand{\blue}{\textcolor{blue}}

\usepackage{versions}
\excludeversion{solution}

\newcommand{\bs}{\begin{solution}}
\newcommand{\es}{\end{solution}}



\begin{document}
\vspace{-1.2in}
\begin{center} \textbf{\Large{Skill Mastery Quiz 3}} \\
Communicating in Math (MTH 210-01)\\
Winter 2020
\end{center}



\noindent Name: 




\begin{itemize}
	\item[L1-3] 
	Consider the following conditional statement:
	\begin{center}
	If $p$ is a prime number then $p=2$ or $p$ is an odd number.
	\end{center}
	Identify the hypothesis and the conclusion of the conditional satement. 
		\bs\textcolor{blue}{The hypothesis is ``$p$ is a prime number" and the conclusion is ``$p=2$ or $p$ is an odd number".}\end{solution}

	\vspace{.5in}
	
	Assume the above conditional statement is true. Assuming \emph{only} the conditional statement and that a given $p$ is odd what can you conclude (if anything)?  Explain your answer.


	\bs\textcolor{blue}{Given that $p$ is odd we know the conclusion of the conditional statement is true. But, if we look in a truth table, knowing the conclusion is true says nothing about the hypothesis.}\end{solution}

\vfill



\item[L2-3] State the definition of even integer precisely:

An integer $n$ is even provided that...

\bs\textcolor{blue}{ there exists an integer $q$ such that $n=2q$.}\end{solution}

\vfill
Outline a proof that if $x,y\in \Z$ are even integers then $xy$ is even. (Make sure to include key details - like what things are integers.)
 
 \bs\textcolor{blue}{Suppose $x$ is even and $y$ is even. Then there exist integers $a$ and $b$ such that $x=2a$ and $y=2b$. Then $xy=(2a)(2b) = 4ab$ by substitution and algebra. By the distributive property $xy=2(2ab)$. Let $q=2ab$. Note that $q$ is an integer because $a$ and $b$ are integers and the integers are closed under addition. Then $xy=2q$ for the integer $q$ and so $xy$ is an even integer.}\end{solution}
\vfill
\vfill
\vfill
\vfill

\newpage

\item[L3-2] Construct a truth table for $P \implies (Q\wedge R)$.

\bs
\begin{center}
\begin{tabular}{c|c|c|c|c}
$P$ &$Q$ &$R$ &$Q\wedge R$ &$P\rightarrow (Q\wedge R)$\\
\hline
T &T &T  &T &T \\
T &F &T  &F &F \\
F &T &T  &T &T \\
F &F &T  &F &T \\
T &T &F &F&F \\
T &F &F &F&F \\
F &T &F &F &T \\
F &F &F &F &T\\
\end{tabular}
\end{center}
\end{solution}
\vfill
\vfill
\vfill


\item[L4-2]  Describe what the set $\{x\in\mathbb{R} \mid 3\leq x\leq 5\}$ is in words. Then write what the set is in roster notation or explain why you can not.

\bs \textcolor{blue}{This is the set of all real numbers between 3 and 5 (including 3 and 5). We can not write this set in roster notation since between any two real numbers there are infinitely many real numbers.}
\end{solution}

\vfill

\item[L5-1] Write a useful negation of the following statement:
		\begin{center}
		There exists $x\in\mathbb{Z}$ such that if $y\in \mathbb{Z}$ then $\frac{y}{x}\in\mathbb{Z}$.
		\end{center}
		Useful negations don't start with ``It is not true that..." and avoid the word not in cases where it could be replaced (e.g., don't use ``not even").

\bs \textcolor{blue}{The negation is: ``for all $x\in\mathbb{Z}$, $y\in\mathbb{Z}$ and $\frac{y}{x}\notin\mathbb{Z}$.}
\end{solution}

\vfill

\end{itemize}
	

\end{document}
