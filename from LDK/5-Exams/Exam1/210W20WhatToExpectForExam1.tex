\documentclass[12pt]{amsart}

\usepackage[margin=.7in]{geometry}

\usepackage{amsmath}

\usepackage{enumerate}
\usepackage{todonotes}
\usepackage{hyperref}
\newtheorem{thm}{Theorem}
\newtheorem{cor}{Corollary}
%\newtheorem{rmk}[thm]{Remark}
\newenvironment{sol}{\paragraph{{\bf Solution:}}}{\hfill$\square$}
\newenvironment{act}{\paragraph{{\bf Activity:}}}{\hfill$\square$}
\usepackage{bm}



\theoremstyle{definition}
\newtheorem*{defn}{Definition}
\newtheorem*{ex}{Example}
\newtheorem*{nonex}{Non-example}
\newtheorem*{rmk}{Remark}

\usepackage{mathtools}


\usepackage{tikz}


\newcommand{\Q}{\mathbb{Q}}
\newcommand{\N}{\mathbb{N}}

\title{What to Expect for Exam 1
}


\begin{document}
MTH 210 - Winter 2020 \hfill Dr. Keough\\

\maketitle

\vspace{-.3in}

\section{Structure }
The exam will be in class on Wednesday February 12. You will have the full 110 minutes. The exam is closed book and notes.  No technology is allowed and no collaboration is allowed. The exam covers Sections 1.1-3.2 (up through the material covered on Wednesday, February 5).

\section{Grading}

The exam will be organized into 4 sections (see the information Section 3 of this document). Each section will be graded as E(xemplary), S(atisfactory), P(rogressing), I(ncomplete). Here is a description of what each grade means:

\begin{center}
\begin{tabular}{|c|c|p{4in}|}
\hline
Letter &Meaning &Description\\
\hline
E &Exemplary &There are no mistakes or errors in the work, including errors in logic and notation, apart from a small number of trivial errors. All solutions are backed up by work that is well written, clearly shown, and easy to read.\\
\hline
S &Satisfactory &Understanding of the concepts is evident, although there may be a small number of mistakes in the work that do not significantly affect the overall reasoning. All solutions are backed up by work that is clearly shown and easy to read.\\
\hline
P &Progressing &Partial understanding of concepts is evident, and all parts of the solution have a good-faith attempt. But there are significant errors in computation, logic, or writing that affect the overall reasoning. The solution should be revised.\\
\hline
I &Incomplete &There are significant omissions in the work, such as parts of the solution left blank intentionally or essential parts of the reasoning behind the answers that are missing or unreadable. Not enough information to determine if the concepts are fully understood. \\
\hline
\end{tabular}
\end{center}
\hspace{.5in}

The grade on the exam will be either E, S, or P+, or P according to the following table 

\begin{center}
\begin{tabular}{|c|c|c|}
\hline
Letter &Meaning &Description\\
\hline
E &Exemplary &All sections earned an S or a E including at least one E. \\
\hline
S &Satisfactory &All sections earned an S.\\
\hline
P+ &Progressing Plus &Earned at least 2 S's or E's, but not all sections earned S's or E's.\\
\hline
P &Progressing &Did not earn at least 2 S's or E's.\\
\hline
\end{tabular}
\end{center}

\newpage

\section{Sections}

The sections are listed below. Each section will have multiple questions, the letter grade will be based on your answers to all questions.
\vfill

\begin{enumerate}
\item {\bf Definitions and Notation} In this section  you may be asked to precisely state the definitions of any of the following terms: even integer, odd integer, type 0/1/2 integer, divides, and congruence. You could be asked to give examples and non-examples of each of the terms or to apply the definitions to explain something (such as why $x\mid 0$ for any nonzero $x$). You also need to know the notations for sets (both set builder and roster), how to translate between them, and how to correctly use $\in$, $\subset$, $\subseteq$, and $=$ (equality of sets).
\vfill
\item {\bf  Logic} For this section you need to know the truth tables for $\vee$ (or), $\wedge$ (and), $\rightarrow$ (implies), $\neg$ (not), and how to combine these into larger truth tables. You could be asked how to apply a conditional statement, to determine truth values based on given information, to use given information to logically deduce new information  (see page 2 of the 2.1/2.2 worksheet). You should know how to negate statements (especially conditional statements and those involving $\forall$ or $\exists$). You should be able to determine the truth values of statements involving $\forall$ or $\exists$. You should know the negation, contrapositive, and converse of a conditional statement and which ones are logically equivalent or have the opposite truth value.
\vfill
\item {\bf Proofs and Counterexamples}  In this section you could be asked to state what you should assume and what you should try to prove in a direct proof, a proof by contrapositive, or a proof of a biconditional statement . You could be asked to find a counterexample to a given statement and explain why it's a counterexample. 
\vfill

\item {\bf Proof Section} \emph{ There will be 2 proofs on the exam.} One of them will be from the following list:
\begin{enumerate}
\item For all nonzero integers $a,b,$ and $c$, if $a\mid b$ and $b\mid c$ then $a\mid c$.

\item For all $a,b,c,d\in\mathbb{Z}$ and all $n\in\mathbb{N}$, if $a\equiv b \pmod{n}$ and $c\equiv d\pmod{n}$ then $ac\equiv bd\pmod{n}$.


\item For all $n\in\mathbb{Z}$, $n$ is even if and only if $4\mid n^2$.



\item If $m\equiv 1 \pmod{3}$ then $3m^2+7m+12 \equiv 1 \pmod{3}$.

\end{enumerate}
 If you do the proofs above in advance then you can hand in the one I ask for on the exam. You may collaborate on these proofs and you can ask me about them. (Make sure you are collaborating appropriately - see the academic honesty document for details. In particular, you should not copy answers, share any electronic files, or just tell someone how to do the proof).\\
 
\noindent The other theorem you'll be asked to prove will be given to you on the exam.\\
  
\noindent \emph{ For both proofs you will need to write according to our writing guidelines.}
\vfill
\end{enumerate}

\newpage

\section{Revisions}

\noindent To revise your exam you should, on separate paper, 
\begin{itemize}
\item redo each question from each section you would like to revise. 
\item write 1-2 sentences explaining the mistake that was made on the exam
\end{itemize}
This task should be done individually (no collaboration is allowed), but you may use your notes and text (no internet sources though). You should then email Dr. Keough (keoulaur@gvsu.edu) to schedule a revision. I will not make appointments in person, you must email me. In your email you should  
\begin{itemize}
\item indicate what sections you are revising
\item attach pictures of your revised work
\item tell me at least 3 different half hours you are available to meet. To aid in this process, my typical weekly schedule is at \url{http://gvsu.edu/s/1gc}
\end{itemize}

When we meet I will ask you questions on your revised and ask you to do additional related questions on the same topic (without your book or notes). So when you prepare for the meeting, you should make sure to go back and study the whole topic, and make sure you understand your revised solutions and the topic as best as you can. Your revised score will be based on both your revised answers and the topic as a whole. Your revision grade will be based on both your revised answers and our meeting.\\

\noindent\textbf{You will get your exam back on Monday, February 17. You must email to \emph{schedule} your appointment by Friday, February 21 at 11:59PM and you must have the appointment before Friday, February 28 at 5PM.}

\vfill

\section{How to Study}

Here are some recommendations for how to study:
\begin{itemize}
\item Go through the self quiz questions on Blackboard. Look through your notes for ones you can not answer.
\item Go through class worksheets, preview activities, synthesis activities, and mastery quizzes and write a summary (focusing on making connections between topics).  
\item Practice!  Look at the extra practice questions on Blackboard, try synthesis activities again (without looking at your solutions), and do exercises on worksheets.  Use the book to find extra exercises on things you are struggling with.
\item Make flashcards for important definitions and notation. See the lists in each of the sections for what you should know.
\item Make flashcards of important tools for negating statements. Know how to negate conditional statements!! And other statements too, but I am definitely going to ask you to negate a conditional statement.
\item Make a list or flash cards for what you assume and what you try to show (e.g., what goes on the top/bottom of a know-show table) for direct proofs, proof by contrapositive, and proofs of biconditional statements. 
\item Take last semester's exam which is posted on Blackboard. It is best to create a testing environment - sit down in a quiet space without any resources and see how much you can do. Note this exam had a very different format and you should not expect this semester's exam to be the same.
\item Write the test! Write yourself a practice exam that you think I would write.  Then take the practice exam.  Feel free to share this with me and I will tell you if you are forgetting big ideas.
\end{itemize}

\end{document}