\documentclass[12pt]{amsart}

\usepackage[margin=.8in]{geometry}

\usepackage{amsmath}

\usepackage{enumerate}
\usepackage{todonotes}
\usepackage{hyperref}
\newtheorem{thm}{Theorem}
\newtheorem{cor}{Corollary}
%\newtheorem{rmk}[thm]{Remark}
\newenvironment{sol}{\paragraph{{\bf Solution:}}}{\hfill$\square$}
\newenvironment{act}{\paragraph{{\bf Activity:}}}{\hfill$\square$}
\usepackage{bm}



\theoremstyle{definition}
\newtheorem*{defn}{Definition}
\newtheorem*{ex}{Example}
\newtheorem*{nonex}{Non-example}
\newtheorem*{rmk}{Remark}

\usepackage{mathtools}


\usepackage{tikz}


\newcommand{\Q}{\mathbb{Q}}
\newcommand{\N}{\mathbb{N}}
\newcommand{\R}{\mathbb{R}}
\newcommand{\Z}{\mathbb{Z}}

\title{What to Expect for Final Exam
}


\begin{document}
MTH 210 - Winter 2020 \hfill Dr. Keough\\

\maketitle


\vspace{-.3in}

The purpose of the final exam is to give you an opportunity to demonstrate all that you have learned in the course.

\section{Grading}
The final exam is out of 100 points. There are 10 questions worth 10 points each. To give you some idea of what points mean, consider an ``S" on a quiz to be at least an 8 out of 10. If you score an 85 or better, a $+$ will be added to your final grade. If you score a 50 or less on the final exam a $-$ will be added to your grade. You also can add a $+$ to your grade if you complete the proof portfolio at the next highest level.\footnote{If you get two +'s your grade goes up by 2/3 of a letter grade, that is, we'll say a ``B$++$" is an A$-$. Also $+$/$-$'s cancel so a ``B$+-$" is a B.} There are no revisions on the final exam and limited feedback will be given. See the syllabus at \url{http://gvsu.edu/s/1kE} for how grades are computed and please ask if you are not sure.

\section{Timing and Resource Use}
Please read the following carefully:
\begin{itemize}
\item The exam will be posted on Blackboard on Monday, April 20 at 10AM. You should upload a single PDF of your answers by Wednesday, April 22, 12PM (the end of our final exam time). 
\item There is no time limit on the exam, but my intention is that it will not take you more time than if we took the exam in class, so plan to spend at most 2 hours. You may want to take advantage of the take home nature of this exam by reading the exam and answering the questions you know for sure and then giving yourself time to walk away. You do not have to take the exam in one sitting.
\item The exam is open book and notes.  You may use a calculator or Desmos (or other graphing technology) if you'd like to graph something, but otherwise no technology is allowed. You should not use Blackboard or other internet resources, and no collaboration is allowed. 
\end{itemize}

\section{How to Study}

While you are allowed to use your book and your notes on the exam, it's still a good idea to prepare for the exam. Here are some recommendations for how to study:
\begin{itemize}
\item Work on the given proofs in (10) - take advantage of being able to talk through those proofs with a classmate.
\item Organize your notes! Make a summary sheet for each of the 10 questions of information you'd like to be able to find easily for that question.
\item Go through the self quiz questions on Blackboard. Look through your notes for ones you can not answer and ask me if you can't find them.
\item Practice!  Look at the extra practice questions on Blackboard, try synthesis activities again (without looking at your solutions), and do exercises on worksheets.  Use the book to find extra exercises on things you are struggling with.
\item Write the test! Write yourself a practice exam that you think I would write.  Then take the practice exam.  
\end{itemize}
\newpage


\section{Material}
The exam is cumulative, covering the sections we covered in Chapters 1-7 of the text. There will be one question about each of the following topics:

\begin{enumerate}
\item \emph{\bf Definitions and Notation:} Be able to know and apply the important definitions from the class, including even integer, odd integer, divides, congruence, rational number, irrational number, function, injection, surjection, bijection, relation, and equivalence relation.
\item \emph{\bf Logic:} This section includes understanding the meanings of and, or, conditional statements, and biconditional statements. You may be asked for a truth table, to determine if conditional statements are true or false based on given information, or to explain logical equivalences. You may be asked to negate a statement.
\item \emph{\bf Proof Techniques - what to assume:} This question will be similar to Section 3, Question 1 on Exam 1.
\item \emph{\bf Proof Techniques - which to use:} This question will be similar to Section 3 on Exam 2.
\item \emph{\bf Counterexamples:} You'll be given a statement, asked to disprove it, and explain why this shows the statement is false. Similar to Skill P3.
\item \emph{\bf Proof Critique:} Similar to Skill P4, and the ``proof evaluation" exercises on the synthesis. You'll be given a proof and asked to determine if it is a valid proof and for any writing guideline violations. You do not need to rewrite the proof. There's lots of examples of these, typically exercise 18 or 19 in the sections in Chapter 3, Exercise 18 in Section 4.1, Exercise 17 in 5.3, Exercise 17 in Section 6.3, Exercise 16 in Section 6.2.
\item \emph{\bf Sets:} Properly use the notation $\subset$, $\subseteq$, $\in$, $=$, $\not\subset$, $\not\subseteq$, $\notin$, $=$. Be able to read and write in both roster and set builder notation. Given sets $A$ and $B$ and a universal set $U$ be able to find $A\cap B$, $A\cup B$, $A-B$, $A^c$, $\mathcal{P}(A)$ (the power set of $A$), $A\times B$, $|A|$ (the cardinality of $A$, and combinations of these things. For sets $A$ and $B$, know how to prove $A\subseteq B$, $A\subset B$, $A=B$, $A\neq B$, and $A\cap B = \emptyset$ ($A$ and $B$ disjoint).
\item \emph{\bf Functions:} You may be asked for examples and non-examples for functions, injections, surjections, and bijections. You may be asked to use the terms domain, codomain, range, image, and preimage, and be able to use them in sentences about a given function. Know how to explain a function is an injection, surjection, or bijection, or not, using the formal definitions.
\item \emph{\bf Equivalence Relations:} You will either be given a relation and asked to determine and explain if it is reflexive, symmetric, and transitive or asked to give an example that is some combination of the properties.
\item \emph{\bf Proof:} I will ask you to prove one of the following two things:
	\begin{enumerate}
	\item Define $g:\R\times\R \to\R$ by $g(a,b) = a$. Then $g$ is a surjection and $g$ is not an injection.
	\item For $(w,x)$ and $(y,z)$ in $\Z\times \Z$ define $(w,x)\sim (y,z)$ if and only if $w\leq y$ and $x\leq z$. Then $\sim$ is reflexive and transitive, but not symmetric.
	\end{enumerate}
	
\end{enumerate}
\vspace{.5in}

\emph{There is lots of opportunity for partial credit here. In particular, if you are not sure how to answer the question given, please explain what you do know. On each question, for up to 7 out of 10 points on that question you can provide some evidence that you understand the topics related to that given question. As an example, if you wrote a practice test you could give that with your solution, or you could give your summary sheet.}

\end{document}