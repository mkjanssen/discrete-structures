\documentclass[11pt]{article}		% The percent symbol in your code starts a comment.  The comment ends at the next linebreak.

\usepackage[english]{babel} 		% Packages add functionality and style conventions to your documents. Don't edit this section!
\usepackage{fullpage}				% Eliminates wasted space
\usepackage[utf8]{inputenc}			% Necessary for character encoding
\usepackage{amsmath, amssymb,amsthm}% Required math packages
\usepackage{graphicx}				% For handling graphics
\usepackage[colorinlistoftodos]{todonotes}	% For the fancy "todo" stuff
\usepackage{hyperref}				% For clickable links in the final PDF
\usepackage{titling}				% To take less space at the top of the page with the title
\setlength{\droptitle}{-2cm}
\pretitle{\begin{flushright}\Large\scshape}
\posttitle{\par\end{flushright}}
\preauthor{\begin{flushright}\large\scshape}
\postauthor{\par\end{flushright}}
\predate{\begin{flushright}\large\scshape}
\postdate{\par\end{flushright}}
\linespread{1.5}

\newcommand{\set}[1]{\left\{ {#1} \right\}}
\newcommand{\setof}[2]{{\left\{#1\,\colon\,#2\right\}}}

% Type `\C' for the complex numbers, `\H' for the quarternions, etc.
\def\C{{\mathbb C}}
\def\H{{\mathbb H}}
\def\Z{{\mathbb Z}}
\def\Q{{\mathbb Q}}
\def\R{{\mathbb R}}
\def\N{{\mathbb N}}


%\Alpha{homeworkresults}

\theoremstyle{definition}
\newtheorem{theorem}{Theorem}
\renewcommand*{\thetheorem}{\Alph{theorem}}
%\setcounter{theorem}{3}
\newtheorem{conj}[theorem]{Conjecture}
\newtheorem{lemma}[theorem]{Lemma}
\newtheorem{prop}[theorem]{Proposition}
\newtheorem{claim}[theorem]{Claim}
\newtheorem{example}[theorem]{Example}



\title{Math 212 Homework 1}

\author{Your name goes here}

\date{Due January 26, 2022}


\begin{document}



\maketitle

\noindent\textbf{Acknowledgments:} \emph{List the names of anyone with whom you discussed these problems \textbf{at all}. Leaving any name off the list constitutes academic dishonesty.}


\begin{example}
	Let $P$ and $Q$ be statements.
	Construct a truth table for the statement
	\[
		[P \land (P \land (P\land Q) \land Q)] \lor Q.
	\]
\end{example}

\noindent\textbf{Solution.}


\begin{table}[htdp]
\begin{center}
\def\arraystretch{1.5}
\begin{tabular}{c|c|c|c|c|c|c}

$P$ & $Q$ & $P\land Q$ & $P\land (P\land Q)$ & $P \land (P\land Q) \land Q$ & $P \land (P \land (P\land Q) \land Q)$ & $[P \land (P \land (P\land Q) \land Q)] \lor Q$ \\\hline

? & ? & ? & ? & ? & ? & ? \\
? & ? & ? & ? & ? & ? & ? \\
? & ? & ? & ? & ? & ? & ? \\
? & ? & ? & ? & ? & ? & ? \\

\end{tabular}
\end{center}
\end{table}%


\begin{example}
	Let $S$ and $T$ be statements.
	Formulate a statement logically equivalent to $S\land (T\lor \neg S)$ using only $S$, $T$, $\neg$, and $\lor$ (each of which you can use as many times as you want/need).
	Explain how you arrived at your equivalent statement (possibly referring to results from Section 1.2), and then use a truth table to prove that your statement is logically equivalent. 
	(Feel free to copy/modify the $\LaTeX$ code from Example A.)
\end{example}


\noindent\textbf{Solution.} \textit{Your solution goes here.}





\begin{example}
	Let $A$ and $B$ be statements.
	Determine whether $(A\lor (\neg A \land B))$ and $\neg [ \neg A \land (A\lor \neg B)]$ are logically equivalent.
	Justify your answer with a truth table that includes at least eight relevant columns.
\end{example}

\noindent\textbf{Solution.} \textit{Your solution goes here.}



\end{document}