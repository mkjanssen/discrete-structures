\documentclass[11pt]{amsart}
\usepackage{amsmath,amssymb,amsthm}
\usepackage{geometry}                % See geometry.pdf to learn the layout options. There are lots.
\geometry{letterpaper}                   % \ldots or a4paper or a5paper or \ldots
%\geometry{landscape}                % Activate for for rotated page geometry
%\usepackage[parfill]{parskip}    % Activate to begin paragraphs with an empty line rather than an indent
\usepackage{graphicx}
\usepackage{amssymb}
\usepackage{epstopdf}
\usepackage{verbatim}
\usepackage{tikz}
\usepackage{xypic}
\usepackage{imakeidx}
\makeindex[columns=2]
\usepackage{hyperref}
\usepackage{enumerate}
\DeclareGraphicsRule{.tif}{png}{.png}{`convert #1 `dirname #1`/`basename #1 .tif`.png}
\usepackage{color}





\hypersetup{
    colorlinks=true,
    linkcolor=blue,
    filecolor=magenta,      
    urlcolor=cyan,
}




\newenvironment{hw}{ \vspace*{7 pt}{\large \textbf{Work for next time}} \vspace*{-6 pt} \\  \vspace*{2 pt}

 {\em In this section, we strive to understand the ideas generated by the following important questions:}
\vspace*{-5 pt} \\  \vspace*{-3 pt}
\begin{list}{\labelitemi}{\leftmargin=1.25em}
\setlength{\itemsep}{3 pt}}{ \end{list} \vspace*{-2 pt}}






\def\card{{\operatorname{card}}}





%% Shortcuts for Math 304 Learning Objectives
%% mathematical practice objectives





%\usepackage{cmbright}

%% A nice way to TeX sets
	\def\set#1{\left\{ {#1} \right\}}
	\def\setof#1#2{{\left\{#1\,\middle\rvert\,#2\right\}}}
	\setlength{\textheight}{8.7truein}
\setlength{\textwidth}{6.5truein}
\setlength{\evensidemargin}{0truein}
\setlength{\oddsidemargin}{0truein}
\setlength{\topmargin}{0truein}
	
%% Some useful macros
	\def\Spec{\operatorname{Spec}}
	\def\Char{\operatorname{char}}	
	\def\C{{\mathbb C}}
	\def\Z{{\mathbb Z}}
	%\def\F{{\mathbb F}}
	\def\bF{{\mathbb F}}
	\def\Q{{\mathbb Q}}
	\def\R{{\mathbb R}}
	\def\P{{\mathbb P}}
	\def\X{{\mathbb X}}
	\def\A{{\mathbb A}}
	\def\V{{\tilde{V}}}
	\def\mP{{\mathcal P}}
	\def\H{{\mathbb H}}
	\def\L{{\mathbb L}}
	\def\N{{\mathbb N}}
	\def\M{{M}}
	\def\sM{{M}}
	\def\B{{\mathcal B}}
	\def\E{{\mathcal E}}
	\def\cC{{\mathcal C}}
	\def\cD{{\mathcal D}}
	\def\O{{\mathcal O}}
	\def\I{{\mathcal I}}
	\def\sC{{\mathscr C}}
	\def\x{{\bold x}}
	\def\c{{1}}
	%\def\l{{\ell}}
	\def\y{{\bold y}}
	\def\field{{k}}
	\def\e{{\varepsilon}}
	\def\Span{{\text{span}}}
	\def\sing{{\text{Sing}}}
	\def\Ass{{\text{Ass}}}
	\def\satdeg{{\text{satdeg}}}
	\def\sat{{\text{sat}}}
	\def\nand{{\overline{\land}}}
	\def\reg{{\text{reg}}}
	\def\dom{{\text{dom}\,}}
	\def\im{{\text{im}\,}}
	\def\ran{{\text{ran}\,}}
	\def\isomorphic{{\,\cong\,}}
	\def\red{\color{red}}

% Injective Arrow
\newcommand{\inj}{\hookrightarrow}
% Named Injective Arrow
\newcommand{\nminj}[1]{\stackrel{#1}{\inj}}

% Surjective Arrow
\newcommand{\surj}{\twoheadrightarrow}
% Named Surjective Arrow
\newcommand{\nmsurj}[1]{\stackrel{#1}{\surj}}


\renewcommand{\emptyset}{\ensuremath\varnothing}

%% custom environments


\newcommand{\ba}{\begin{enumerate}[(a)]}
\newcommand{\ea}{\end{enumerate}}



\newcommand{\bi}{\begin{enumerate}[(i)]}
\newcommand{\ei}{\end{enumerate}}

%% Setting up theorems

\theoremstyle{plain}

\newtheorem{thm}{Theorem}
\newtheorem{cor}[thm]{Corollary}
\newtheorem{prop}[thm]{Proposition}
\newtheorem{lem}[thm]{Lemma}
\newtheorem{claim}[thm]{Claim}


\theoremstyle{definition}



\newtheorem{example}[thm]{Example}
\newtheorem{prob}[thm]{Problem}
\newtheorem{ex}[thm]{Example}
\newtheorem{exdefn}[thm]{Example/Definition}
%\newtheorem{defn}[thm]{Definition}
\newtheorem{fact}[thm]{Fact}
\newtheorem{conj}[thm]{Conjecture}
%\newtheorem{question}[thm]{Question}
%\newtheorem{problem}[thm]{Problem}
\newtheorem{notation}[]{Notation}
\newtheorem{rem}[thm]{Remark}
\newtheorem{axiom}{Axiom}
\newtheorem*{homework}{\red Work for next time}
%\newtheorem{exer}{{\color{red} Exercise}}[section]

\newtheorem{innerproblem}{Problem}
\newenvironment{problem}[1]
  {\renewcommand\theinnerproblem{#1}\innerproblem}
  {\endinnerproblem}

\newtheorem{innerquestion}{Question}
\newenvironment{question}[1]
  {\renewcommand\theinnerquestion{#1}\innerquestion}
  {\endinnerquestion}

\newtheorem{innerexercise}{Exercise}
\newenvironment{exercise}[1]
  {\renewcommand\theinnerexercise{#1}\innerexercise}
  {\endinnerexercise}
  
\newtheorem{innerstatement}{Statement}
\newenvironment{statement}[1]
  {\renewcommand\theinnerstatement{#1}\innerstatement}
  {\endinnerstatement}
  
  
  
\newtheorem{innerdefinition}{Definition}
\newenvironment{defn}[1]
  {\renewcommand\theinnerdefinition{#1}\innerdefinition}
  {\endinnerdefinition}
  
  
  
\newtheorem{innerexploration}{Exploration}
\newenvironment{expl}[1]
  {\renewcommand\theinnerexploration{#1}\innerexploration}
  {\endinnerexploration}
  
  
  
  
\newtheorem{innerexample}{Example}
\newenvironment{eg}[1]
  {\renewcommand\theinnerexample{#1}\innerexample}
  {\endinnerexample}
  

  
  
%% more readable font on an iPad
%\usepackage{mathpple}
%\usepackage{baskervald}
%\usepackage[bigdelims,vvarbb]{newtxmath} 

%% Highlight questions I want to ask in class
\def\q#1{{\color{red} {#1}}}


\def\mysection#1{\renewcommand{\thesection}{Meeting \arabic{section}}
\section{#1}
\renewcommand{\thesection}{\arabic{section}}}

%% black square for QED
%\renewenvironment{proof}[1][\proofname]{\begin{trivlist}\pushQED{\qed}\item[\hskip \labelsep  \bfseries #1{}.\hspace{10pt}]}
%{\popQED\end{trivlist}}
%\renewcommand{\qedsymbol}{$\blacksquare$}



\title{Portfolio 0: Introduction to \LaTeX}
\author{Dr.~Mike Janssen}
\address{Department of Mathematics and Statistics, Dordt University, Sioux Center IA 51250}
\email{mike.janssen@dordt.edu}

\date{}                                           % Activate to display a given date or no date

\def\presnotes{

\noindent  \textbf{Presenter name:} \hfill \textbf{Score:} \makebox[0.5in]{\hrulefill} \ \textbf{Date:} \makebox[0.5in]{\hrulefill}

\ \\

\noindent  \textbf{Scribe:} 

\ \\

\noindent \textbf{Notes:}

\ \\
}


\begin{document}



\maketitle


\section{Introduction}

%One of the strengths of our textbook is that it aids in the learning of \LaTeX, the typesetting language in which v
Virtually all mathematics (and many works in computer science and the natural sciences) are produced using the \LaTeX\ document preparation system. \LaTeX\ is not particularly hard to learn, but it will take a little bit of time before you are completely comfortable.

You will write your written homework in \LaTeX. This is a precondition for passing the assignment! Having a typed version of your work will also help you revise your it based on initial feedback.

%This semester, you are being asked to write your homework solutions in \LaTeX, which is the \textit{lingua franca} of mathematics. Simply put, this is how mathematicians write mathematics, and it will be a valuable experience regardless of where you are called after college. However, I recognize that there will be a learning curve associated with \LaTeX, and probably/possibly some frustration as well. 
Here is my best advice for completing homework using \LaTeX:
\begin{itemize}\renewcommand{\labelitemi}{$\diamond$}
	\item \textbf{Solve the problem before writing it up!} In other words, scratch paper is your friend. Work out the details in as messy a fashion as you want, but take care to write things in a logical way, as though you were writing a short essay.
	\item \textbf{Google is your friend when you can't find that symbol you want!} If you can think of something you want to do in \LaTeX, chances are it can be done, and someone has written about it on the Internet somewhere.  The number of times I've searched something like ``{\tt latex blah symbol}'' is nigh uncountable.
	%\item Appendix C of our textbook is worth referring to, as well.
	\item This guide is helpful, too: {\tt \href{https://prof.mkjanssen.org/ds/LaTeX/latexcheatsheet.pdf}{https://prof.mkjanssen.org/ds/LaTeX/latexcheatsheet.pdf}}
	\item Detexify is useful for determining the command for a symbol: 
	
	{\tt \href{http://detexify.kirelabs.org/classify.html}{http://detexify.kirelabs.org/classify.html}}
	\item \textbf{Google is your friend when you're presented with a compiler error!} I would guess that approximately 95\% of the \LaTeX\ compiler errors you will see this semester will be the result of a missing `\}' or using a command/environment that is undefined (or misspelled). But for all other errors, try Google first.
	\item \textbf{\LaTeX\ helpfully ignores most whitespace!} Hitting a carriage return once after a period is the same as pressing space after a period. It is good form to put every sentence on a new line, as the errors will refer to a line number, and this will cut down the area you need to search when presented with an error.
	\item \textbf{Send me an email!} One advantage of using \LaTeX\ is that it gives us a way to discuss mathematics (or \LaTeX\ itself) via email, so if you can't find what you're looking for on Google, make sure to send me your questions. 
\end{itemize}
%Finally, take advantage of Dr.~Robert Talbert's excellent \href{https://www.youtube.com/playlist?list=PLF975D9D3C9B50FF7}{YouTube playlist} on writing in \LaTeX.


\section{The assignment}

\subsection{Getting Started}

\noindent\textbf{Instructions.} Some of the exercises that follow have multiple parts. In the \LaTeX\ document you create in Exercise 1, you should create one new section per exercise, following the Introduction, which you should rename to `Exercise 1' e.g., \verb!\section{Exercise 2}!, etc. I have done my best to \textbf{bold} text which describes something that should go into your final document. When you are done, download the PDF and submit it on Canvas.

There are several ways you can write in \LaTeX, all of them free.

\begin{enumerate}
	\item By far the easiest choice is Overleaf, found at {\tt \href{http://overleaf.com}{http://overleaf.com}}, which you can think of as Google Docs for \LaTeX. \textbf{You should register for an account there.}   %Our course notes will be hosted at {\tt \href{http://bit.ly/m207s20}{http://bit.ly/m207s20}}, so you'll get plenty of practice with it.
	\item Alternatively, if you have a Mac, Mac\TeX\ is great: {\tt \href{http://www.tug.org/mactex/}{http://www.tug.org/mactex/}}. The downside is that the download and installation take up a lot of space on your drive, but the upside is that you no longer need internet access to use \LaTeX. Mac\TeX\ comes with the {\TeX}Shop editor.
	\item If you are using Windows, the usual choice of engine is MiK\TeX: {\tt \href{https://miktex.org}{https://miktex.org}}. {\TeX}nicCenter is a solid choice for an editor.
\end{enumerate}

\noindent \textbf{Exercise 1.} Create a new document. If you are using Overleaf, create a blank document. You should end up with something like this:

\begin{verbatim}
\documentclass{article}
\usepackage[utf8]{inputenc}

\title{Intro to LaTeX}
\author{Mike Janssen}
\date{January 2019}

\begin{document}

\maketitle

\section{Introduction}

\end{document}
\end{verbatim}

\textbf{Rename the Introduction section to be `Exercise 1'.}

\textbf{Type ``Hello, world'' into the Introduction section.}

\noindent\textbf{Exercise 2.} This blank document is nice and simple, but we want to add features to our document that makes it easier to produce mathematical symbols.

\begin{enumerate}[(a)]
	\item One of the major strengths of \LaTeX\ is that it allows you to easily insert mathematical symbols in either \emph{inline math mode} or \emph{display math mode}. 
	
	Inline math mode refers to the insertion of mathematical symbols and expressions in the middle of a sentence/paragraph. It is done between dollar signs. For example, typing: 

	\begin{verbatim}``The Fundamental Theorem of Calculus says, roughly, that $\int_a^b f'(x) dx =
	 f(b) - f(a)$''\end{verbatim} 
	 
	 yields ``The Fundamental Theorem of Calculus says, roughly, that $\int_a^b f'(x) dx = f(b) - f(a)$''.
	 
	 \medskip  
	
	Display math mode makes some symbols (such as summations or integrals) a bit larger, and puts the expression on its own line. (When typing in display math mode, it's helpful to give the code its own line, too.) There are two options for display math mode: double dollar signs, or slashes and square brackets, as follows: 
	
	\begin{verbatim}The Fundamental Theorem of Calculus says, roughly, that 
	\[
		\int_a^b f'(x) dx = f(b) - f(a)
	\]\end{verbatim}
	
	This yields:
	
	``The Fundamental Theorem of Calculus says, roughly, that 
	\[
		\int_a^b f'(x) dx = f(b) - f(a)''
	\]
	
	\textbf{Now}, figure out how to typeset the limit defintion of the derivative function, which I'll show in display math mode:
	
	\[
		f'(x) = \lim\limits_{h\to 0} \frac{f(x+h)-f(x)}{h}.
	\]
	
	\textbf{Do the FTC and limit definition of the derivative look better in display math mode or inline? Why do you answer the way you do?}
	
	%\textbf{First, read Section C.5 (pp.~140--141 of the text). Then attempt to typset the limit definition of the derivative on p.~141 in both inline and display math modes.} Which looks better? [Make sure to recompile when you've made a change to your document!]
	\item Eventually we will need symbols not accessible to our simple document; for example, $\Z$ is the standard notation for the set of all integers. \textbf{Try typesetting} \verb!$\mathbb{Z}$!. 
	What happens?
	\item We can extend the functionality of our document by including new packages in the preamble (the part before \verb!\begin{document}!). \textbf{In the line after} \verb!\usepackage[utf8]{inputenc}!, type \verb!\usepackage{amssymb}! and recompile. Now what happens?}
\end{enumerate}



\noindent\textbf{Exercise 3.} In order to make your solutions and proofs more readable, it is often advisable to move Very Important Steps into the center of the page. There are multiple ways to do this. When you just have one line, using display math mode is easiest: \verb! \[ math goes here \]!. If you have a single line you'd like to refer to later, using the equation environment is best.

\begin{verbatim}
\begin{equation}\label{Equation:IBP}
	\int u\, dv = uv - \int v\, du.
\end{equation}
\end{verbatim}

\begin{enumerate}[(a)]
	\item \textbf{Reproduce the equation environment from above.} Then, in a line after the equation, type \verb!\ref{Equation:IBP}! and recompile. What do you notice?
\end{enumerate}

A final reason you might want to center math in the page is to show a series of steps. You can do this with the {\tt array} environment or the {\tt align*} environment:

\begin{verbatim}
\begin{align*}
	1 + 2 + \cdots + k + (k+1) &= \frac{k(k+1)}{2} + k + 1\\
	&= \frac{k^2 +k}{2} + \frac{2k+2}{2} \\
	&= \frac{k^2 + 3k + 2}{2} \\
	&= \frac{(k+1)(k+2)}{2} \\
	&= \frac{(k+1)((k+1)+1)}{2}.
\end{align*}
\end{verbatim}

Here, the lines are aligned along the ampersands, and the double backslash \verb!\\! creates a new line.

But oh! What error do we see? Let's fix it by adding the {\tt amsmath} package to our preamble (the part before \verb!\begin{document}!.)

\begin{enumerate}[(a)]
	\item[(b)] \textbf{Try typesetting the {\tt align*} example above. After recompiling, remove the asterisk from the opening and closing {\tt align} commands. What happens?}
%	\item[(c)] Which do you like better? Pick your favorite and typeset it into the document.
\end{enumerate}


\noindent\textbf{Exercise 4.} As described above, \LaTeX\ helpfully ignores most white space (spaces and carriage returns). 

\begin{enumerate}[(a)]
	\item Rather than putting spaces between sentences, you can hit return. \textbf{Type the following into your document.}
	\begin{verbatim}
	For He must reign until He has put all His enemies under His feet. The last
	enemy to be destroyed is death.
	
	For He must reign until He has put all His enemies under His feet.
	The last enemy to be destroyed is death.
	\end{verbatim}
	The advantage to putting new sentences on new lines is that, when the compiler detects an error, it typically reports a line number. It's much easier to scan a sentence for an error than a paragraph.
	\item The best way to create a new line in \LaTeX\ is by putting a blank line into the text document. \textbf{Type the following into your document.}

\begin{verbatim}
But Christ has indeed been raised from the dead, the firstfruits of those 
who have fallen asleep. For since death came through a man, the resurrection 
of the dead comes also through a man. For as in Adam all die, so in Christ 
all will be made alive. But each in his own turn: Christ the firstfruits; 
then at His coming, those who belong to Him.

Then the end will come, when He hands over the kingdom to God the Father after 
He has destroyed all dominion, authority, and power. For He must reign until He 
has put all His enemies under His feet. The last enemy to be destroyed is death.
\end{verbatim}

Let me emphasize: \emph{this is the desired way for you to start new paragraphs.} Do not use forced new lines unless absolutely necessary.
\end{enumerate}

 

%\noindent\textbf{Exercise 5.} Early this semester, you will need to typset some tables. Use the example of the truth table for $P\wedge Q$ at the bottom of p.~145 as a model, and \textbf{typeset the truth table for $P\Rightarrow Q$}, which can be found near the bottom of p.~23.


\noindent\textbf{Exercise 5.} You will likely make several \LaTeX\ errors as you learn to use it. Here are two of the most common.

\begin{enumerate}[(a)]
	\item Many \LaTeX\ commands involve typing curly braces, \{ and \}. A very common error is to forget the closing curly brace. \textbf{Try typing the following into your document. Then, replace the faulty command with the type of error message you receive.}
	\begin{verbatim}
	$ \frac{\int x\, dx}{2 $
	\end{verbatim}
	Notice that we have forgotten the right curly brace on the denominator of our fraction.
	%\newpage
	\item Another common error is to forget to start math mode. \textbf{Try typing}
	\begin{verbatim}
	\frac{\int x\, dx}{2}.
	\end{verbatim}
	\textbf{In the document, replace the faulty command with the type of error message you receive.}
\end{enumerate}



\noindent\textbf{Exercise 6.} \LaTeX\ is a \emph{typesetting} language, meaning that your job is to provide the text of your document and use commands to tell \LaTeX what the various types of text are. It worries about how to format and present them. We have seen that it will typeset text/commands in math mode as mathematical expressions, and that it ignores lots of whitespace. The last big idea we'll explore is that of an \emph{environment}. Environments typically begin with \verb!\begin{!\emph{environment name}\verb!}! and end with \verb!\end{!\emph{environment name}\verb!}!.

\begin{enumerate}[(a)]
	\item The {\tt enumerate} and {\tt itemize} environments are good for making lists. Use the following as a guide, and \textbf{make a grocery list with at least four items on it} (you choose the items):
	\begin{verbatim}
	\begin{enumerate}
		\item Your first item goes here
	\end{enumerate}
	\end{verbatim}
	\item \textbf{Make a copy of your list, but change {\tt enumerate} to {\tt itemize}}. What happens?
	\item We will use the {\tt theorem} and {\tt proof} environments to contain our theorems and proofs. Our document gained support for the theorem environment when we added the {\tt amsmath} package, but we need to add the {\tt amsthm} package to get the proof environment. Now, create a new theorem in the preamble with \verb!\newtheorem{theorem}{Theorem}!. 
	
	\newpage
	
	With the help of your resources (cheat sheet, internet, etc.), type the following exactly as it appears:
\end{enumerate}

\medskip

		\begin{thm}
			For all natural numbers $n$, $n^2 + 3n+2$ is even.
		\end{thm}
		
		\begin{proof}
			Let $n$ be a natural number.
			Observe that
			\[
				n^2 + 3n + 2 = (n+1)(n+2).
			\]
			We notice that $n+1$ and $n+2$ are consecutive integers, and by Statement 2.18, their product is even.
			Therefore, $n^2 + 3n + 2$ is even.
		\end{proof}
		
		
\noindent\textbf{Exercise 7.} Finally, read over the \LaTeX\ cheat sheet found at

\begin{center}
 {\tt \href{https://prof.mkjanssen.org/ds/LaTeX/latexcheatsheet.pdf}{https://prof.mkjanssen.org/ds/LaTeX/latexcheatsheet.pdf}}}.\end{center} 
 
 \textbf{What questions do you have?}

		
		There is, of course, a lot more to learn. My hope is that this gives you a solid foundation for writing your portfolio.

\end{document}
